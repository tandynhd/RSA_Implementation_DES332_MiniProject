% This document is written by Hung and Terrayut. Some part is borrowed from EC template (http://www2.siit.tu.ac.th/somsak/SeniorProjects/2015/2015_SeniorProjectReportTemplate.zip)
\documentclass[12pt, a4paper]{report}
\usepackage[top=1in,bottom=1in,left=1.25in,right=1in]{geometry}
\usepackage{setspace}
\usepackage{graphicx}
\usepackage{times}
\usepackage{subcaption,url,hyperref}
\usepackage{listings}
\usepackage{amsmath}

% put all figures in foder ./figure
\graphicspath{{./figure/}}


\begin{document}

%%%% BEGIN: TITLE PAGE %%%%
\begin{center}

\large 
\vspace*{2cm}

MINI PROJECT FOR COMPUTE AND NETWORK SECURITY\\[2cm]

\LARGE


Here is the name of your system\\[2cm]
\large
Submitted to \\[1cm]
School of Information, Computer and Communication Technology \\
Sirindhorn International Institute of Technology \\
Thammasat University \\[2cm]
May 2020 \\[3cm]
by \\[1cm]
StudentName1   StuID1 \\
StudentName2   StuID2\\[2cm]
\end{center}
%%%% END: TITLE PAGE %%%%


\newpage
\pagestyle{plain}
\pagenumbering{roman}
\onehalfspace

\chapter*{Abstract} 
\addcontentsline{toc}{chapter}{Abstract}

A short paragraph summarizing the provided features of the implemented system.

%%%% BEGIN: TABLE OF CONTENT %%%%
\addcontentsline{toc}{chapter}{Contents} 
\tableofcontents
%%%% END: TABLE OF CONTENT %%%%


%%%% BEGIN: FIGURES %%%%
\addcontentsline{toc}{chapter}{List of Figures} 
\listoffigures
%%%% END: FIGURES %%%%


\newpage
\thispagestyle{empty}
\pagenumbering{arabic}
\setcounter{chapter}{0}
\setcounter{section}{0}
\def\thechapter{\arabic{chapter}}

%%%% BEGIN: CHAPTER 1: INTRODUCTION %%%%
%\input{5-introduction}

\chapter{Project Concept}

In this chapter, you should write a good and concise introduction to your project. 

\section{Summary}
\label{sec:summary}
What is done in the project?  For example:

\begin{itemize}
	\item Implement RSA algorithms (key generation and enc/decr)
	
	\item Use the implemented RSA algorithms and 3DES from the library xxx to provide two security services for email:
	
			\begin{itemize}
				\item authentication: verify that received email really came from the sender.
				
				\item confidentiality: protection from disclosure (no one but the receiver can successfully decrypt the encrypted email) 
			\end{itemize}
\end{itemize}




\section{Typical Usage}
\label{sec:usage}

How will the system be used in a typical scenario? What are some of the key features? 


\section{Main Challenges}
\label{sec:challenges}

What is  the hardest and/or most time consuming part of the project? 

\chapter{Project requirements}


\section{System description}

Authentication of an email as having come from a specific sender follows these steps:

	\begin{enumerate}
	
		\item The sender creates a clear-text message.
		
		\item The sender creates a SHA-1 message digest of the clear-text message.
		
		\item The sender encrypts the SHA-1 message digest using the RSA asymmetric encryption algorithm with the sender’s private key, producing a digital signature that is attached to the clear-text message.
		
		\item A receiver uses the RSA asymmetric encryption algorithm with the sender's public key to decrypt the digital signature and recover the SHA-1 message digest.
		
		\item The receiver generates a SHA-1 message digest from the clear-tex tmessage and compares the generated SHA-1 message digest with the decrypted SHA-1 message digest; if they match, then the message is accepted as authentic.
	
	
	\end{enumerate}
	
Confidentiality protection of a message follows these steps:

	\begin{enumerate}
		\item The sender generates a random 128-bit number to be used as a session shared secret key (SSSK) for this message only.
		\item The sender encrypts the clear-text message, and appended digital signature,using a symmetric encryption algorithm, such as CAST-128, IDEA or 3DES, with the SSSK.
		\item The sender then encrypts the SSSK using RSA with each recipient’s public key(s) and then appends each uniquely encrypted copy of the SSSK to the black-text message.
		\item Each receiver uses RSA with its private key to decrypt and recover their copy of the SSSK.
		\item The decrypted SSSK is used to decrypt the black-text message thereby recovering the clear-text message.
	\end{enumerate}
\section{Computational tasks}


		\begin{enumerate}
				\item RSA algorithms
				
						\begin{enumerate}
							\item Key generation
							
									\begin{enumerate}
										\item Prime number generation
										
										\item Computing ...
									\end{enumerate}
							
							\item Encryption
							
									\begin{enumerate}
										\item Modular exponentiation
										
										\item Encrypting a text/string
									
									\end{enumerate}
						\end{enumerate}
						
				\item 
		\end{enumerate}

\section{Use cases}


\chapter{Algorithm design and Implementation}

\section{Algorithm design}

For each computational task, describe your algorithm to solve the task.

\section{Implementation}

Describe how you implemented the presented algorithms. You can write pseudo code or some code snippets.

%%%% start of appendices %%%%
\appendix
\chapter{Computing modular exponentiation}

\lstinputlisting[language=Python]{./code/modularExp.py}


\end{document}